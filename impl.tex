\section{Implementation}\label{sec:implementation}

%$$$$$$$$$$$$$$$$$$$$$$$$$$$$$$$$$$$$$$$$$$$$$$$$$$$$$$$$$$$$$$$$$$$$$$$$$$$$$$$$
%Paragraph 1: 커널 버전 및 코드 분량 설명
%$$$$$$$$$$$$$$$$$$$$$$$$$$$$$$$$$$$$$$$$$$$$$$$$$$$$$$$$$$$$$$$$$$$$$$$$$$$$$$$$



%$$$$$$$$$$$$$$$$$$$$$$$$$$$$$$$$$$$$$$$$$$$$$$$$$$$$$$$$$$$$$$$$$$$$$$$$$$$$$$$$
%Paragraph 2: 커널 데이터 구조에 추가한 필드 설명
%$$$$$$$$$$$$$$$$$$$$$$$$$$$$$$$$$$$$$$$$$$$$$$$$$$$$$$$$$$$$$$$$$$$$$$$$$$$$$$$$



%$$$$$$$$$$$$$$$$$$$$$$$$$$$$$$$$$$$$$$$$$$$$$$$$$$$$$$$$$$$$$$$$$$$$$$$$$$$$$$$$
%Paragraph 3: lock을 제거한 부분에 대한 설명 
%$$$$$$$$$$$$$$$$$$$$$$$$$$$$$$$$$$$$$$$$$$$$$$$$$$$$$$$$$$$$$$$$$$$$$$$$$$$$$$$$




%$$$$$$$$$$$$$$$$$$$$$$$$$$$$$$$$$$$$$$$$$$$$$$$$$$$$$$$$$$$$$$$$$$$$$$$$$$$$$$$$
%Paragraph 4: object를 나중에 free하는 내용 설명 
%$$$$$$$$$$$$$$$$$$$$$$$$$$$$$$$$$$$$$$$$$$$$$$$$$$$$$$$$$$$$$$$$$$$$$$$$$$$$$$$$
% 기존 코드를 변경없이 





%$$$$$$$$$$$$$$$$$$$$$$$$$$$$$$$$$$$$$$$$$$$$$$$$$$$$$$$$$$$$$$$$$$$$$$$$$$$$$$$$
%$$$$$$$$$$$$$$$$$$$$$$$$$$$$$$$$$$$$$$$$$$$$$$$$$$$$$$$$$$$$$$$$$$$$$$$$$$$$$$$$
%Reference Sentence 1
%$$$$$$$$$$$$$$$$$$$$$$$$$$$$$$$$$$$$$$$$$$$$$$$$$$$$$$$$$$$$$$$$$$$$$$$$$$$$$$$$





%$$$$$$$$$$$$$$$$$$$$$$$$$$$$$$$$$$$$$$$$$$$$$$$$$$$$$$$$$$$$$$$$$$$$$$$$$$$$$$$$
%Reference Sentence 2:LDU paper
%$$$$$$$$$$$$$$$$$$$$$$$$$$$$$$$$$$$$$$$$$$$$$$$$$$$$$$$$$$$$$$$$$$$$$$$$$$$$$$$$
%\section{Implementation}\label{sec:implementation}
%We implemented the new deferred update algorithm in Linux 3.19.rc4 kernel, and
%our modified Linux is available as open source.
%\deferu's scheme is based on deferred processing, so it needs a garbage
%collector for delayed free.
%In order to implement the garbage collector, we use the lock-less list and a
%periodic timer(1 sec) in the Linux.

%Paragraph 2: 문제점을 해결하기 위해 Harris linked list를 적용
%We compare our \deferu implementation to a concurrent non-blocking Harris
%linked list ~\cite{Harris2001Lockfree};therefore, we implement the Harris
%% linked list to Linux kernel.
%The code refers from sysnchrobench~\cite{Gramoli2015Synchrobench} and
%ASCYLIB~\cite{David2015ASYNCHRONIZED}, and we convert their linked list to
%Linux kernel style.
%Because both synchrobench and ASCYLIB leak memory, we implement additional
%garbage collector for the Linux kernel using Linux's work queues and lock-less
%list.

%Paragraph 3: 오브젝트의 특징을 고려한 lock-free list 구현 
%In order to further improve performance, we move their ordered list to
%unordered list. 
%A feature of the Harris linked list is all the nodes are ordered by
%their key. 
%Zhang~\cite{zhang2013practical} implements a lock-free unordered list
%algorithm, whose list is each insert and remove operation appends an
%intermediate node at the head of the list;these approach is practically
%hard to implement.
%Indeed, Linux does not require contains operation because the Linux data
%structures such as list, tree and hash table not depended on search key;they
%depend on their unique object.
%This feature can eliminates the ordered list in Harris linked list.
%Therefore, we perform each insert operation appends an intermediate node at
%the first node of the list;on the other hand, each remove operation searches
%from head to their node.

%Paragraph 4: mapping은 DeferU와 Harris linked 리스트 둘 다 적용, 
%하지만 anon은 Harris Lined list 만 적용과 이유
%To the scalability of fork, the reverse mapping's lock contention should
%be eliminated not only from file reverse mapping but also from anonymous
% reverse mapping.
%The structure of file reverse mapping is simplified relatively to the
%structure of anonymous mapping because the anonymous reverse mapping is
%entangled by their global object(\code{anon\_vma}) and their
%chain(\code{anon\_vma\_chain});therefore, we only apply \deferu to file reverse
%mapping.

