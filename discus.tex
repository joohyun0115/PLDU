\section{Discussion and future work}

%$$$$$$$$$$$$$$$$$$$$$$$$$$$$$$$$$$$$$$$$$$$$$$$$$$$$$$$$$$$$$$$$$$$$$$$$$$$$$$$$
%Paragraph 1: 더 General하게 사용되도록 만들어야 한다.
%$$$$$$$$$$$$$$$$$$$$$$$$$$$$$$$$$$$$$$$$$$$$$$$$$$$$$$$$$$$$$$$$$$$$$$$$$$$$$$$$
우리는 오직 high update rate를 가진 data structure를 대상으로 LDU를 구현되었다. 
하지만 실제 high update rate를 가진 data structure는 워크로드 마다 상황이 틀리다.
앞에서 설명한 두가지 reverse mapping도 역시 fork intencive 워크로드이여야지 high update rate를 가진
data structure라고 말할 수 있다. 
결국 log-based approach가 low update rate를 가질때는 오히려 성능이 떨어질 수 있다.
그러므로 log-based approach인 LDU를 high update rate을 가질 때만 적용할 수 있도록 해야 한다.
본 논문은 이 부분에 대해서는 고려하지 않았다.
따라서 임베디드 시스템등 다양한 분야에 사용되는 리눅스 커널에 적용되기 위해서는 update rate를 판단하여 동작하도록 해야지 보다
범용적으로 high update rate 상황과 general학

%$$$$$$$$$$$$$$$$$$$$$$$$$$$$$$$$$$$$$$$$$$$$$$$$$$$$$$$$$$$$$$$$$$$$$$$$$$$$$$$$
%Paragraph 2: reader 쪽을 개선해야 한다. 
%$$$$$$$$$$$$$$$$$$$$$$$$$$$$$$$$$$$$$$$$$$$$$$$$$$$$$$$$$$$$$$$$$$$$$$$$$$$$$$$$

현재 LDU는 data structure를 보호하기 위해 exclucive lock를 사용하도록 구현하였다. 
하지만 concurrent read를 위해 concurrent update와 concurrent read를 고려해서 구현해야 한다.
이 방법은 





%$$$$$$$$$$$$$$$$$$$$$$$$$$$$$$$$$$$$$$$$$$$$$$$$$$$$$$$$$$$$$$$$$$$$$$$$$$$$$$$$
%$$$$$$$$$$$$$$$$$$$$$$$$$$$$$$$$$$$$$$$$$$$$$$$$$$$$$$$$$$$$$$$$$$$$$$$$$$$$$$$$
%Reference Sentence 1
%$$$$$$$$$$$$$$$$$$$$$$$$$$$$$$$$$$$$$$$$$$$$$$$$$$$$$$$$$$$$$$$$$$$$$$$$$$$$$$$$





%$$$$$$$$$$$$$$$$$$$$$$$$$$$$$$$$$$$$$$$$$$$$$$$$$$$$$$$$$$$$$$$$$$$$$$$$$$$$$$$$
%Reference Sentence 2:LDU paper
%$$$$$$$$$$$$$$$$$$$$$$$$$$$$$$$$$$$$$$$$$$$$$$$$$$$$$$$$$$$$$$$$$$$$$$$$$$$$$$$$

