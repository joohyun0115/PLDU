\section{Discussion and future work}

%$$$$$$$$$$$$$$$$$$$$$$$$$$$$$$$$$$$$$$$$$$$$$$$$$$$$$$$$$$$$$$$$$$$$$$$$$$$$$$$$
%Paragraph 1: 더 범용적이게 만들어야 한다.
%$$$$$$$$$$$$$$$$$$$$$$$$$$$$$$$$$$$$$$$$$$$$$$$$$$$$$$$$$$$$$$$$$$$$$$$$$$$$$$$$
\ifkor
우리는 hight update rate data structure만 고려 하였지만, LDU는 보다 범용적인 상황에 적용 수 있게 만들어져야
한다.
그 이유는 실제 high update rate를 가진 상황은 자주 발생되는 상황이 아니다. 
따라서 data structure의 operation의 update rate이 적을 경우에는 불필요하게 log를 저장하므로 오히려 성능이 떨어
지는 역효과가 생긴다.
이것은 임베디드 시스템등 다양한 분야에 사용되는 리눅스 커널이 때문에 문제가 있다. 
따라서 항상 log-based로 처리하지 않고 update rate이 증가하면, 판단하여 log-based로 처리하도록 해야 한다.
즉 high update rate과 상관없이 좋은 성능을 보이는 범용적인 알고르즘으로 발전 시켜야한다.
\else

\fi

%$$$$$$$$$$$$$$$$$$$$$$$$$$$$$$$$$$$$$$$$$$$$$$$$$$$$$$$$$$$$$$$$$$$$$$$$$$$$$$$$
%Paragraph 2: reader 쪽을 개선해야 한다. 
%$$$$$$$$$$$$$$$$$$$$$$$$$$$$$$$$$$$$$$$$$$$$$$$$$$$$$$$$$$$$$$$$$$$$$$$$$$$$$$$$

\ifkor
현재 LDU는 data structure를 보호하기 위해 exclucive lock를 사용하도록 구현하였다. 
하지만 concurrent read를 위해 concurrent update와 concurrent read를 고려해서 구현해야 한다.
\else
\fi






%$$$$$$$$$$$$$$$$$$$$$$$$$$$$$$$$$$$$$$$$$$$$$$$$$$$$$$$$$$$$$$$$$$$$$$$$$$$$$$$$
%$$$$$$$$$$$$$$$$$$$$$$$$$$$$$$$$$$$$$$$$$$$$$$$$$$$$$$$$$$$$$$$$$$$$$$$$$$$$$$$$
%Reference Sentence 1
%$$$$$$$$$$$$$$$$$$$$$$$$$$$$$$$$$$$$$$$$$$$$$$$$$$$$$$$$$$$$$$$$$$$$$$$$$$$$$$$$





%$$$$$$$$$$$$$$$$$$$$$$$$$$$$$$$$$$$$$$$$$$$$$$$$$$$$$$$$$$$$$$$$$$$$$$$$$$$$$$$$
%Reference Sentence 2:LDU paper
%$$$$$$$$$$$$$$$$$$$$$$$$$$$$$$$$$$$$$$$$$$$$$$$$$$$$$$$$$$$$$$$$$$$$$$$$$$$$$$$$

