\section{Discussion}

%$$$$$$$$$$$$$$$$$$$$$$$$$$$$$$$$$$$$$$$$$$$$$$$$$$$$$$$$$$$$$$$$$$$$$$$$$$$$$$$$
%Paragraph 1: 더 범용적이게 만들어야 한다.
%$$$$$$$$$$$$$$$$$$$$$$$$$$$$$$$$$$$$$$$$$$$$$$$$$$$$$$$$$$$$$$$$$$$$$$$$$$$$$$$$
\ifkor
우리는 리눅스 커널과 같이 update operation이 insert-insert또는 remove-remove와 같이 같은 operation에
대해서 발생하지 않는 data structure의 경우에 대해서만 사용 가능하도록 LDU를 구현하였다.
이러한 방법은 굉장히 practical한 방법으로 리눅스와 freeBSD와 같은 커널에는 적용 할 수 있으나 
보다 연구 중심적인 data structure인 CSDS 알고리즘과 비교 실험하는데는 무리가 있다. 
따라서 LDU 또는 Oplog와 같이 log-based 방법을 보다 연구 중심적인 data structure와 비교 실험이
가능하도록 보다 일반화할 필요가 있다. 
\else

\fi

%$$$$$$$$$$$$$$$$$$$$$$$$$$$$$$$$$$$$$$$$$$$$$$$$$$$$$$$$$$$$$$$$$$$$$$$$$$$$$$$$
%$$$$$$$$$$$$$$$$$$$$$$$$$$$$$$$$$$$$$$$$$$$$$$$$$$$$$$$$$$$$$$$$$$$$$$$$$$$$$$$$
%Reference Sentence 1
%$$$$$$$$$$$$$$$$$$$$$$$$$$$$$$$$$$$$$$$$$$$$$$$$$$$$$$$$$$$$$$$$$$$$$$$$$$$$$$$$





%$$$$$$$$$$$$$$$$$$$$$$$$$$$$$$$$$$$$$$$$$$$$$$$$$$$$$$$$$$$$$$$$$$$$$$$$$$$$$$$$
%Reference Sentence 2:LDU paper
%$$$$$$$$$$$$$$$$$$$$$$$$$$$$$$$$$$$$$$$$$$$$$$$$$$$$$$$$$$$$$$$$$$$$$$$$$$$$$$$$

