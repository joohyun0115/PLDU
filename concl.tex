\section{Conclusion}


\section{Acknowledgments}


%$$$$$$$$$$$$$$$$$$$$$$$$$$$$$$$$$$$$$$$$$$$$$$$$$$$$$$$$$$$$$$$$$$$$$$$$$$$$$$$$
%Reference Sentence 1
%$$$$$$$$$$$$$$$$$$$$$$$$$$$$$$$$$$$$$$$$$$$$$$$$$$$$$$$$$$$$$$$$$$$$$$$$$$$$$$$$


%$$$$$$$$$$$$$$$$$$$$$$$$$$$$$$$$$$$$$$$$$$$$$$$$$$$$$$$$$$$$$$$$$$$$$$$$$$$$$$$$
%Reference Sentence 2
%$$$$$$$$$$$$$$$$$$$$$$$$$$$$$$$$$$$$$$$$$$$$$$$$$$$$$$$$$$$$$$$$$$$$$$$$$$$$$$$$

%We propose a concurrent update algorithm, \deferu, for update-heavy data
%structures scalable for many-core systems.
%To achieve the scalability during process spawning, 
%we applied deferred log processing with global log queue and 
%update-side absorbing to Linux reverse mapping.
%Evaluation results using the AIM7, Exim and lmbench reveal that \deferu shows
%better performance up to 2.2 times compared to existing solutions.
%\deferu is implemented on to Linux kernel 3.19 and available as open-source
% from \url{https://github.com/KMU-embedded/scalablelinux}.

%\section{Acknowledgments}
%This work was supported by Institute for Information \& communications
% Technology Promotion (IITP) grant funded by the Korea government (MSIP) (14-824-09-
%011, “Research Project on High Performance and Scalable Manycore Operating
% System”)

