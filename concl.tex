\section{Conclusion and Discussion}
We eliminated the synchronized timestamp counters by using the LDU, and
we applied the LDU to Linux reverse mapping in order to achieve the scalability
during process spawning.
Evaluation results reveal that \deferu shows better performance up to 2.7 times
stock linux kernel on the 120 core machine.

To achieve our goal, this paper only focus on the eliminating the synchronized
timestamp counters.
However, although the LDU eliminate time-sensitive log by using update-side
removing, the remain logs can not apply some data structure such as stack and
queue(section x).
One mightily solution is that combine two techniques(LDU and synchronized
timestamp counters) in order to allow this limited data structure to support.

\deferu is implemented on to Linux kernel 4.5 and available as open-source
 from \url{https://github.com/manycore-ldu/ldu}.

%\section{Acknowledgments}
%This work was supported by Institute for Information \& communications
% Technology Promotion (IITP) grant funded by the Korea government (MSIP) (14-824-09-
%011, “Research Project on High Performance and Scalable Manycore Operating
% System”)

