\section{Introduction} \label{sec:introduction}

%$$$$$$$$$$$$$$$$$$$$$$$$$$$$$$$$$$$$$$$$$$$$$$$$$$$$$$$$$$$$$$$$$$$$$$$$$$$$$$$$
%$$$$$$$$$$$$$$$$$$$$$$$$$$$$$$$$$$$$$$$$$$$$$$$$$$$$$$$$$$$$$$$$$$$$$$$$$$$$$$$$
%Background
%$$$$$$$$$$$$$$$$$$$$$$$$$$$$$$$$$$$$$$$$$$$$$$$$$$$$$$$$$$$$$$$$$$$$$$$$$$$$$$$$
\ifkor
최근 코어수가 증가하고 있다. 따라서 멀티코어에서 매니코어 시스템으로 바뀌고 있다. 
매니코어 시스템에 대한 운영체제 커널의 parallelism은 시스템 전체의 parallelism에서 가장 중요하다. 
만약 커널이 scale하지 않으면, 그 위에 동작하는 응용프로그램들도 역시 scale하지 않는다[].
이처럼 중요한 운영체제 커널 중 멀티코어 또는 매니코어 환경에서 많이 사용되는 운영체제가 리눅스 커널이다[].
하지만 리눅스 커널은 아직 확장성 문제가
있다~\cite{SilasBoydWickizer2010LinuxScales48}~\cite{Changwoo2016UMSF}.
확장성 문제 중 하나는 락 경합 때문에 발생하는 업데이트 직렬화
문제이다~\cite{Matveev2015RLU}~\cite{Dodds2015SCT}.
그 이유는 업데이트 오퍼레이션은 여러 쓰레드가 동시에 수행되지 못하기
때문이다~\cite{mckenney2011parallel}.
\else
%최근 number of core가 증가하고 있다. 따라서 multicore system에서 manycore system시스템으로 바뀌고
% 있다.
%manycore system에 대한 operationg system kernel의 parallelism은 시스템 전체의
% parallelism에서 가장 중요하다.
%만약 kernel이 scale하지 않으면, 그 위에 동작하는 application들도 역시 scale하지 않는다[].
%이처럼 중요한 operating system kernel 중 multicore or manycore 환경에서 많이
% 사용되는 operating system가 linux kernel이다[].
%하지만 Linux kernel은 아직 scalability problem이
%있다~\cite{SilasBoydWickizer2010LinuxScales48}~\cite{Changwoo2016UMSF}.
%scalability problem 중 하나는 lock contention 때문에 발생하는 update serialization 
%problem이다~\cite{Matveev2015RLU}~\cite{Dodds2015SCT}.
%그 이유는 update operation은 여러 thread 동시에 수행되지 못하기
%때문이다~\cite{mckenney2011parallel}.
\fi


%$$$$$$$$$$$$$$$$$$$$$$$$$$$$$$$$$$$$$$$$$$$$$$$$$$$$$$$$$$$$$$$$$$$$$$$$$$$$$$$$
%$$$$$$$$$$$$$$$$$$$$$$$$$$$$$$$$$$$$$$$$$$$$$$$$$$$$$$$$$$$$$$$$$$$$$$$$$$$$$$$$
%Problem gernal
%$$$$$$$$$$$$$$$$$$$$$$$$$$$$$$$$$$$$$$$$$$$$$$$$$$$$$$$$$$$$$$$$$$$$$$$$$$$$$$$$
\ifkor
이처럼 업데이트 직렬화 문제를 해결하기 위해 여러 동시적 업데이트 방법 들이 연구되고
있다~\cite{Arbel2014ConcurrentRCU}~\cite{Matveev2015RLU}.
이러한 동시적 업데이트 방법들은 워크로드 특성인 업데이트 비율에 따라 많은 성능 차이를
보인다~\cite{Matveev2015RLU}.
이 중 높은 업데이트 비율을 가진 자료 구조 때문에 발생하는 학장성 문제를 해결하기 위한 여러 방법이 연구 되고
있다.
그 중 하나는 cache communication bottleneck을 줄인 log-based
알고리즘~\cite{Shalev2006PLS}~\cite{Hendler2010FC}~\cite{SilasBoydWickizerPth}을
사용하는 것이다.
Log-based 알고리즘은 업데이트가 발생하면, data structure의 업데이트 operation을
per-core 또는 atomic하게 log로 저장하고 read operation을 수행하기 전에 저장된 로그를 수행하는것이다 .
이것은 마치 CoW(Copy On Write)와 유사하다~\cite{PaulDetailLWN}.
\else
%이처럼 update serialization 문제를 해결하기 위해 여러 concurrent updates
%방법 들이 연구되고 있다~\cite{Arbel2014ConcurrentRCU}~\cite{Matveev2015RLU}.
%이러한 concurrent updates 방법들은 워크로드 특성인 update ratio에 따라 많은 성능 차이를
%보인다~\cite{Matveev2015RLU}.
%이 중 high update rate을 가진 data stgructure 때문에 발생하는 scalability problem를 해결하기 위한
% 여러 방법이 연구 되고 있다.
%그 중 하나는 cache coherence traffic을 줄인 log-based
%알고리즘~\cite{Shalev2006PLS}~\cite{Hendler2010FC}~\cite{SilasBoydWickizerPth}을
%사용하는 것이다.
%Log-based 알고리즘은 update가 발생하면, data structure의 update operation을
%per-core 또는 atomic하게 log로 저장하고 read operation을 수행하기 전에 저장된 log를 수행하는것이다 .
%이것은 마치 CoW(Copy On Write)와 유사하다~\cite{PaulDetailLWN}.
\fi

%$$$$$$$$$$$$$$$$$$$$$$$$$$$$$$$$$$$$$$$$$$$$$$$$$$$$$$$$$$$$$$$$$$$$$$$$$$$$$$$$
%$$$$$$$$$$$$$$$$$$$$$$$$$$$$$$$$$$$$$$$$$$$$$$$$$$$$$$$$$$$$$$$$$$$$$$$$$$$$$$$$
%Problem 
%$$$$$$$$$$$$$$$$$$$$$$$$$$$$$$$$$$$$$$$$$$$$$$$$$$$$$$$$$$$$$$$$$$$$$$$$$$$$$$$$
\ifkor
S. Boyd-Wickizer et al.는 동기화된 타임스탬프 카운터(synchronized timestamp counters) 기반의
per-core log를 활용하여 update-heavy한 자료구조를 대상으로 동시적 업데이트 문제를
해결함과 동시에 cache communication bottleneck을 줄였다~\cite{SilasBoydWickizerPth}.
동기화된 타임스탬프 카운터 기반의 per-core log를 활용한 동시적 업데이트방법은
업데이트 부분만 고려했을 때, per-core에 데이터를 저장함으로 굉장히 높은 scalability를 가진다[].
하지만 per-core 기반의 동기화된 타임스탬프 카운터를 사용한 방법은 결국 timestamp merging and ordering 작업을
야기한다.
만약 코어 수가 늘어 날 경우, 로그를 자료 구조에 적용하는 과정에서 timestamp 때문에 발생하는 추가적인 sequntial 프로세싱이
요구된다.
이것은 결국 확장성과 성능을 저해한다. 
\else
%S. Boyd-Wickizer et al. 는 synchronized timestamp counters 기반의 per-core log를
% 활용하여 update heavy한 data structure를 대상으로 concurrent updates 문제를
%해결함과 동시에 cache communication bottleneck을 줄였다~\cite{SilasBoydWickizerPth}.
%Synchronized timestamp counters 기반의 per-core log를 활용한 concurrent updates 방법은
%update 부분만 고려했을 때, per-core에 데이터를 저장함으로 굉장히 높은 scalability를 가진다.
%하지만 per-core 기반의 synchronized timestamp counters를 사용한 방법은 결국 timestamp
% merging and ordering 작업을 야기한다.
%만약 코어 수가 늘어 날 경우, resolving logs(merging, absorbing) may require
%additional sequential processing, which can limit scalalbility and
%performance.
\fi



%$$$$$$$$$$$$$$$$$$$$$$$$$$$$$$$$$$$$$$$$$$$$$$$$$$$$$$$$$$$$$$$$$$$$$$$$$$$$$$$$
%$$$$$$$$$$$$$$$$$$$$$$$$$$$$$$$$$$$$$$$$$$$$$$$$$$$$$$$$$$$$$$$$$$$$$$$$$$$$$$$$
%Method
%$$$$$$$$$$$$$$$$$$$$$$$$$$$$$$$$$$$$$$$$$$$$$$$$$$$$$$$$$$$$$$$$$$$$$$$$$$$$$$$$
%
\ifkor
본 논문은 동기화된 타임스탬프 카운터를 이용함에 따라 생기는 추가적인 sequencial processing 문제를
해결하기 위해 shared memory system을 위한 새로운 LDU(Lightweitgh log-based Deferred Update)를
개발하였다.
LDU는 타임스탬프 카운터가 필요한 operation log를 업데이트 순간 지우고, 매번 로그를 생성하지 않고 재활용하는 방법이다.
이로인해 synchronized timestamp counter 문제와 cache communication bottleneck 문제를 동시에
해결하였다.
해결 방법은 분산 시스템에서 사용하는 synchronized timestamp log기반의 concurrent updates 방식[]과
최소한의 shared-memory system의 hardware-based synchronization 기법(compare and swap,
test and set, atomic swap)을 조합하여 동시적 업데이트 문제를 해결하였다.
\else
%본 논문은 synchronized timestamp counters를 이용함에 따라 생기는 추가적인 sequencial processing 문제를
%해결하기 위해 shared memory system을 위한 새로운 LDU(Lightweitgh log-based Deferred
% Update)를 개발하였다.
%LDU는 timestamp counters가 필요한 operation log를 update 순간 지우고, 매번 로그를 생성하지 않고 재활용하는
% 방법이다.
%이로인해 synchronized timestamp counter 문제와 cache communication bottleneck 문제를 동시에
%해결하였다.
%해결 방법은 분산 시스템에서 사용하는 synchronized timestamp log기반의 concurrent updates 방식[]과
%최소한의 shared-memory system의 hardware-based synchronization 기법(compare and swap,
%test and set, atomic swap)을 조합하여 동시적 업데이트 문제를 해결하였다.
\fi

\ifkor
이처럼 동기화된 타임스탬프 카운터를 제거함과 동시에, cache communication bottleneck 줄인
LDU는 기존 log-based 알고리즘들의 장점들을 모두 포함할 뿐만아니라 추가적인 장점을 가진다.
첫째로, update가 수행하는 시점 즉 로그를 저장하는 순간에는 lock이 필요가 없다. 
따라서 lock에 대한 오버헤드 없이 concurrent updates를 수행할 수 있다
둘째로, 저장된 update operation log를 coarse-grained lock과 함께 하나의 코어에서 수행하기 때문에,
cache 효율성이 높아진다~\cite{Hendler2010FC}.
셋째로, 기존 여러 자료구조에 쉽게 적용할 수 있는 장점이 있다.
게다가 마지막으로, log를 저장하기 전에 로그를 삭제하므로 보다 빠르게 log의 수를 줄일 수 있다. 
\else
%이처럼 synchronized timestamp counters를 제거함과 동시에, cache communication bottleneck
% 줄인 LDU는 기존 log-based algorithm들의 장점들을 모두 포함할 뿐만아니라 추가적인 장점을 가진다.
%첫째로, update가 수행하는 시점 즉 로그를 저장하는 순간에는 lock이 필요가 없다. 
%따라서 lock에 대한 오버헤드 없이 concurrent updates를 수행할 수 있다
%둘째로, 저장된 update operation log를 coarse-grained lock과 함께 하나의 코어에서 수행하기 때문에,
%cache 효율성이 높아진다~\cite{Hendler2010FC}.
%셋째로, 기존 여러 자료구조에 쉽게 적용할 수 있는 장점이 있다.
%게다가 마지막으로, log를 저장하기 전에 로그를 삭제하므로 보다 빠르게 log의 수를 줄일 수 있다. 
\fi

%$$$$$$$$$$$$$$$$$$$$$$$$$$$$$$$$$$$$$$$$$$$$$$$$$$$$$$$$$$$$$$$$$$$$$$$$$$$$$$$$
%$$$$$$$$$$$$$$$$$$$$$$$$$$$$$$$$$$$$$$$$$$$$$$$$$$$$$$$$$$$$$$$$$$$$$$$$$$$$$$$$
%Result
%$$$$$$$$$$$$$$$$$$$$$$$$$$$$$$$$$$$$$$$$$$$$$$$$$$$$$$$$$$$$$$$$$$$$$$$$$$$$$$$$

\ifkor
우리는 위와 같은 장점을 가지는 LDU를 리눅스 커널에서 high update rate 때문에 scalability 문제를 야기시키는
anonymous reverse mapping과 file reverse mapping에 적용하였다.
또한 우리는 LDU를 Linux 4.5.rc4에 구현하였고, fork-intensive 워크로드인
AIM7~\cite{AIM7Benchmark}, Exim~\cite{Exim} from MOSBENCH~\cite{MOSBENCH},
lmbench~\cite{mcvoy1996lmbench}를 대상으로 성능 개선을 보였다. 개선은 stock 리눅스 커널에 비해 120코어에서
각각 x,x,x 배이다.
\else
%우리는 위와 같은 장점을 가지는 LDU를 리눅스 커널의 fork scalability 문제를 야기시키는 2가지 reverse page
%mapping(anonymous pape mapping, file page mapping)에 적용하였다.
%또한 We implemented the LDU in a Linux 4.5.rc4 with elemination of lock. 
%We evaluated the performance and scalability using a fork-intensive workload-
%AIM7~\cite{AIM7Benchmark}, Exim~\cite{Exim} from MOSBENCH~\cite{MOSBENCH},
%lmbench~\cite{mcvoy1996lmbench}-our design improves throughput and execution
%time on 120 core by x, x, x respectively, relative to stock Linux.
\fi


%$$$$$$$$$$$$$$$$$$$$$$$$$$$$$$$$$$$$$$$$$$$$$$$$$$$$$$$$$$$$$$$$$$$$$$$$$$$$$$$$
%$$$$$$$$$$$$$$$$$$$$$$$$$$$$$$$$$$$$$$$$$$$$$$$$$$$$$$$$$$$$$$$$$$$$$$$$$$$$$$$$
%Contribution 정리
%$$$$$$$$$$$$$$$$$$$$$$$$$$$$$$$$$$$$$$$$$$$$$$$$$$$$$$$$$$$$$$$$$$$$$$$$$$$$$$$$
\ifkor
\noindent
\textbf{Contributions.} This paper makes the following contributions:
\begin{itemize}
\item 우리는 high update rate를 가지는 data structuer를 위한 새로운 log-based
concurrnet updates 방법인 LDU를 개발하였다.
LDU는 동기화된 타임스탬프 카운터를 이용함에 따라 생기는 시간 정렬과 머징에 의한 추가적인 sequencial processing 문제를
최소한의 hardware-based synchronization 기법을 사용하여 해결하였다.
LDU는 hardware-based synchronization 기법을 이용하여 LDU는 로그를 업데이트 순간 지우고 로그를 재활용한다.
\item 우리는 LDU을 practical한 manycore system인 intel xeon 120코어 위에 동작하는 리눅스 커널의
2가지 reverse mapping(anonymous, file)에 적용하여, fork scalability 문제를 해결하였다.
Fork 관련 벤치마크 성능은 워크로드 특성에 따라 1.6x부터 2.2x까지 개선되었다.
\end{itemize}
\else

\textbf{Contributions.} This paper makes the following contributions:
\begin{itemize}
\item high update rate
%\item 우리는 high update rate를 가지는 data structuer를 위한 새로운 log-based
%concurrnet updates 방법인 LDU를 개발하였다.
%LDU는 동기화된 타임스탬프 카운터를 이용함에 따라 생기는 시간 정렬과 머징에 의한 추가적인 sequencial processing 문제를
%최소한의 hardware-based synchronization 기법을 사용하여 해결하였다.
%LDU는 hardware-based synchronization 기법을 이용하여 LDU는 로그를 업데이트 순간 지우고 로그를 재활용한다.
%\item 우리는 LDU을 practical한 manycore system인 intel xeon 120코어 위에 동작하는 리눅스 커널의
%2가지 reverse mapping(anonymous, file)에 적용하여, fork scalability 문제를 해결하였다.
%Fork 관련 벤치마크 성능은 워크로드 특성에 따라 1.6x부터 2.2x까지 개선되었다.
\end{itemize}

\fi


%$$$$$$$$$$$$$$$$$$$$$$$$$$$$$$$$$$$$$$$$$$$$$$$$$$$$$$$$$$$$$$$$$$$$$$$$$$$$$$$$
%$$$$$$$$$$$$$$$$$$$$$$$$$$$$$$$$$$$$$$$$$$$$$$$$$$$$$$$$$$$$$$$$$$$$$$$$$$$$$$$$
%Mapping
%$$$$$$$$$$$$$$$$$$$$$$$$$$$$$$$$$$$$$$$$$$$$$$$$$$$$$$$$$$$$$$$$$$$$$$$$$$$$$$$$
\ifkor
The rest of this paper is organized as follows.
Section 2 describes the background and Linux scalability problem.
Section 3 describes the design of the LDU algorithm and 
Section 4 explains how to apply to Linux kernel.
section 5 explains our implementations in Linux and
Section 6 shows the results of the experimental evaluation. 
Finally, section 8 concludes the paper.
\else
The rest of this paper is organized as follows.
Section 2 describes the background and Linux scalability problem.
Section 3 describes the design of the LDU algorithm and 
Section 4 explains how to apply to Linux kernel.
section 5 explains our implementations in Linux and
Section 6 shows the results of the experimental evaluation. 
Finally, section 8 concludes the paper.
\fi

%$$$$$$$$$$$$$$$$$$$$$$$$$$$$$$$$$$$$$$$$$$$$$$$$$$$$$$$$$$$$$$$$$$$$$$$$$$$$$$$$
%$$$$$$$$$$$$$$$$$$$$$$$$$$$$$$$$$$$$$$$$$$$$$$$$$$$$$$$$$$$$$$$$$$$$$$$$$$$$$$$$
% Reference Sentence 1
%$$$$$$$$$$$$$$$$$$$$$$$$$$$$$$$$$$$$$$$$$$$$$$$$$$$$$$$$$$$$$$$$$$$$$$$$$$$$$$$$
%
%





%$$$$$$$$$$$$$$$$$$$$$$$$$$$$$$$$$$$$$$$$$$$$$$$$$$$$$$$$$$$$$$$$$$$$$$$$$$$$$$$$
%Reference Sentence 2:LDU paper
%$$$$$$$$$$$$$$$$$$$$$$$$$$$$$$$$$$$$$$$$$$$$$$$$$$$$$$$$$$$$$$$$$$$$$$$$$$$$$$$$
% Ondemand로 로그를 지워 준다.
%Background
%With the drastic increase of CPU core counts in various high-end
%server systems, achieving performance scalability of operating systems
%running on such many-core systems has been an important issue in research
%communities.
%Linux has been naturally considered as a major target for the scalability
%improvement and a number of accomplishments are published.
%Early results include RCU~\cite{McKenney98} and hazard
%pointer~\cite{MagedMichael04a} to improve scalability for read-most data 
%structures in the Linux kernel. %for relatively low CPU core counts.
%Though the early results show certain level of improvement in scalability,
%it turned out that more significant portion of scalability limitation of
%Linux kernel is due to lock contention in update-heavy global data structures 
%including file reverse mappings and anonymous reverse mappings during
%spawning child processes~\cite{Andi2011adding}~\cite{Tim2013adding}. 
%Such update-heavy data structures cause serialization of the update operations
%leading to severe performance degradation. 

%Scaling operating systems to many-core architectures is one of the most
%important challenges in computing today. 
%One of the scalable operations system is Linux because the Linux kernel
%community has made it scalable.
%For example, read-mostly data structures in the Linux have been achieved
%considerable multi-core scaling by using RCU~\cite{McKenney98} and hazard
%pointer~\cite{MagedMichael04a}.
%However, the Linux kernel suffers from such scalability bottlenecks at high
%core counts.
%The Linux kernel on many-core processors can be
%bottlenecked by contended updates locks where processes share a global data
%structure.
%When a process spawns child process, reverse mapping using shared global
%data structure suffers from updates lock contention.
%Recent research shows how to run a fork in parallel with creating new reverse
%mapping data structure~\cite{SilasBoydWickizerPth}.
%More specifically, in order to perfect scalability of the fork, both the
%file reverse mapping and the anonymous reverse mapping can execute
%update concurrently without lock
%contention~\cite{Andi2011adding}~\cite{Tim2013adding}.

%The fundamental scalability problem of reverse mapping is their serialized
%update because operating systems are serialized at the update operation.

%To solve this problem, an existing approach is to make the update-heavy data
%structures as non-blocking~\cite{Harris2001Lockfree} based on
% \emph{compare-and-swap}(CAS).
%Introducing non-blocking data structures eliminates the update serialization
% problem during process spawning, but incurs additional issues due to
% inter-core communication
%bottlenecks and cache coherence system's write
% serialization~\cite{SilasBoydWickizerPth}.
%To overcome the issues caused by cache coherence system, S. Boyd-Wickizer et
% al. proposed Oplog~\cite{SilasBoydWickizerPth} where logs update operations
% with time stamps and actual updates are performed later when the updated data
% need to be read.
%While Oplog nicely solves the update serialization problem without any cache
% coherence-related overheads, the merging of the update logs recorded in
% multiple per-core data structures considering time stamps further causes
% performance overheads resulting in limited scalability
% improvement~\cite{McKenney2008ParallelProgramming}.

%Therefore, many researches have proposed non-blocking
%algorithms~\cite{Harris2001Lockfree} for concurrent data structure based on
%\emph{compare-and-swap}(CAS);nonetheless, this method may suffer from inter-core
%communication bottleneck, the cache coherence system serializes the
% writes~\cite{SilasBoydWickizerPth}.

%Another a existing solution is using the per-core processing like
%Oplog~\cite{SilasBoydWickizerPth}, which achieves scalability by logging in
%per-core memory.
%In fact, per-core approach may have best performance for the update-heavy data
%structure because of their partitioned data structures, whose updates can
%operate locally.
%The more-expensive reads, however, must merge across the entire per-core
%data;read operations are expensive~\cite{McKenney2008ParallelProgramming}.
%Therefore, their approach may be complex with regard to the read operation.
%This paper attacks this complexity as a result of the per-core processing.
%Even though they have generalized a per-core processing to the Oplog's library, 
%they use intricate optimizations, so their optimizations may be expensive with
%regard to the read operation.
%For example, they use the absorbing updates that removes the cancelable
%operation before the read.
%This operation may need to iterate previous operation log due to searching the
%cancelable operation log.
%Furthermore, reducing the memory use, they maintain per-core memory space.

%Method
%This paper proposes a novel concurrent update method, \deferu, applicable to
% Linux reverse mapping solving the problems 
%mentioned above: the overheads caused by inter-core communication bottlenecks
% and per-core log management with time stamps. 
%Our goal is to make the Linux fork scale to large numbers of cores using
%lightweight deferred processing algorithm.
%The fork scalability requires two challenges in designing a reverse mapping.
%First, this lightweight method should permit the concurrent updates not only to
%reduce the inter-core communication bottleneck but also to eliminate their 
%complexity.
%Second, this lightweight method should apply to Linux reverse mapping, and
%improve the Linux fork scalability.

%The \deferu is similar to Oplog in that it defers the actual update operations
% as late as possible to reduce serialization problems, but it uses a light
% weight global queue with non-blocking synchronization for update logs and
% eliminates time stamps required for per-core log management. 
%In addition, to optimize the log management and minimize the traversal
% overheads during reading, \deferu applies update-side absorbing algorithm based on atomic
%marking and thus efficiently find the operations to be canceled. 
%The evaluation of the proposed \deferu on Linux kernel 3.19.rc4 running on a
% 120 core system reveals that 
%the execution times could be improved by 1.7x, 1.6x, and 2.2x for 
%a fork-intensive workload-
%AIM7~\cite{AIM7Benchmark}, Exim from
%MOSBENCH~\cite{SilasBoydWickizer2010LinuxScales48}, and 
%lmbench~\cite{mcvoy1996lmbench}, respectively.

%Second, it uses a novel update-side absorbing that uses the atomic marking
%method, which allows \deferu to eliminate read-side traversal for finding the
%cancelable operation log;readers can improve performance.

%The proposed approach has the following advantages. 
%First, it can permits the reverse mapping to remove lock contention, so Linux's
%fork scalability has been improved.
%Second, using the lightweight update-side absorbing, \deferu can reduce
%complexity and improve readers performance.
%Finally, while per-core processing uses the time-stamp counter(e.g., RDTSCP,
%RDTSC) that depend on hardware, our method may not depend on hardware.

%We implemented the \deferu in a Linux 3.19.rc4 with modification of lock. 
%We evaluated the performance and scalability using a fork-intensive workload-
%AIM7~\cite{AIM7Benchmark}, Exim from
%MOSBENCH~\cite{SilasBoydWickizer2010LinuxScales48},
%lmbench~\cite{mcvoy1996lmbench}-our design improves throughput and execution
%time on 120 core by 1.7x, 1.6x, 2.2x respectively, relative to stock Linux.

%Mapping
%Paragraph 
%This paper is organized as follows. 
%Section 2 summarizes related works and compare our contributions to previous
%works. 
%Section 3 describes the design of the \deferu algorithm and 
%Section 4 explains how to apply to Linux kernel.
%section 5 explains our implementations in Linux and
%Section 6 shows the results of the experimental evaluation. 
%Finally, section 7 concludes the paper.