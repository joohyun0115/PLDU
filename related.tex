\section{Related work} \label{sec:RelatedWork}
%$$$$$$$$$$$$$$$$$$$$$$$$$$$$$$$$$$$$$$$$$$$$$$$$$$$$$$$$$$$$$$$$$$$$$$$$$$$$$$$$
%Paragraph 1:Linux Scalability의 연구에 대한 설명
%$$$$$$$$$$$$$$$$$$$$$$$$$$$$$$$$$$$$$$$$$$$$$$$$$$$$$$$$$$$$$$$$$$$$$$$$$$$$$$$$
\noindent
\textbf{Operating system scalability.}
~\cite{Clements15SCR}In order to improve Linux scalability, researchers have
been optimized in Linux by finding and fixing scalability
BonsaiVM~\cite{AustinTClements2012RCUBalancedTrees} solved this address space
problem by using the RCU;
RadixVM~\cite{Clements2013RadixVM} created a new VM using refcache and radix
tree, which enable \code{munmap}, \code{mmap}, and \code{page fault} on
non-overlapping memory regions to scale perfectly.
Alternatively, to avoid contention caused by shared address space locking,
system programmers change their multithreaded applications to use
processes~\cite{SilasBoydWickizer2010LinuxScales48}.

%$$$$$$$$$$$$$$$$$$$$$$$$$$$$$$$$$$$$$$$$$$$$$$$$$$$$$$$$$$$$$$$$$$$$$$$$$$$$$$$$
%Paragraph 2:Concurrent updates에 대한 연구
%$$$$$$$$$$$$$$$$$$$$$$$$$$$$$$$$$$$$$$$$$$$$$$$$$$$$$$$$$$$$$$$$$$$$$$$$$$$$$$$$
\noindent
\textbf{Scalabe data structure.}
~\cite{Dodds2015SCT} Arbel and Attiya~\cite{Arbel2014ConcurrentRCU} shows a new
design of concurrent search tree called the Citrus tree. The Citrus tree
When increasing the update rate, Citrus tree still suffers from bottlenecks.
RLU~\cite{Matveev2015RLU} presents a new synchronization mechanism that allows
unsynchronized sequences of reads to execute concurrently with updates.
In high update rate, Oplog can achieve substantially multi-core scaling for
update-heavy data structures.

%$$$$$$$$$$$$$$$$$$$$$$$$$$$$$$$$$$$$$$$$$$$$$$$$$$$$$$$$$$$$$$$$$$$$$$$$$$$$$$$$
%Paragraph 3: Scalable Data Structure and Lock에 대한 연구
%$$$$$$$$$$$$$$$$$$$$$$$$$$$$$$$$$$$$$$$$$$$$$$$$$$$$$$$$$$$$$$$$$$$$$$$$$$$$$$$$
\noindent
\textbf{Scalable lock.}
~\cite{Wang2016BeMyGuest}~\cite{Bueso2015STP}~\cite{Bueso2014MCS}One method for
the concurrent update is using the non-blocking algorithms~\cite{Harris2001Lockfree}~\cite{Fomitchev2004Lockfree}~\cite{Timnat2012},
 which are based on CAS.
MCS~\cite{MellorCrummey91}, a scalable exclusive lock, is used in the Linux
kernel~\cite{MCSLocksKernel}.
Reader-writer lock~\cite{Courtois71} allows either number of readers to execute
In read-mostly data structures, RCU~\cite{McKenney98} can be quite useful


