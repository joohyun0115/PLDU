\section{related work} \label{sec:RelatedWork}
%$$$$$$$$$$$$$$$$$$$$$$$$$$$$$$$$$$$$$$$$$$$$$$$$$$$$$$$$$$$$$$$$$$$$$$$$$$$$$$$$
%Paragraph 1:Linux Scalability의 연구에 대한 설명
%$$$$$$$$$$$$$$$$$$$$$$$$$$$$$$$$$$$$$$$$$$$$$$$$$$$$$$$$$$$$$$$$$$$$$$$$$$$$$$$$



%$$$$$$$$$$$$$$$$$$$$$$$$$$$$$$$$$$$$$$$$$$$$$$$$$$$$$$$$$$$$$$$$$$$$$$$$$$$$$$$$
%Paragraph 2:Concurrent updates에 대한 연구
%$$$$$$$$$$$$$$$$$$$$$$$$$$$$$$$$$$$$$$$$$$$$$$$$$$$$$$$$$$$$$$$$$$$$$$$$$$$$$$$$



%$$$$$$$$$$$$$$$$$$$$$$$$$$$$$$$$$$$$$$$$$$$$$$$$$$$$$$$$$$$$$$$$$$$$$$$$$$$$$$$$
%Paragraph 3: Scalable Data Structure and Lock에 대한 연구
%$$$$$$$$$$$$$$$$$$$$$$$$$$$$$$$$$$$$$$$$$$$$$$$$$$$$$$$$$$$$$$$$$$$$$$$$$$$$$$$$



%$$$$$$$$$$$$$$$$$$$$$$$$$$$$$$$$$$$$$$$$$$$$$$$$$$$$$$$$$$$$$$$$$$$$$$$$$$$$$$$$
%$$$$$$$$$$$$$$$$$$$$$$$$$$$$$$$$$$$$$$$$$$$$$$$$$$$$$$$$$$$$$$$$$$$$$$$$$$$$$$$$
%Reference Sentence 1
%$$$$$$$$$$$$$$$$$$$$$$$$$$$$$$$$$$$$$$$$$$$$$$$$$$$$$$$$$$$$$$$$$$$$$$$$$$$$$$$$




%$$$$$$$$$$$$$$$$$$$$$$$$$$$$$$$$$$$$$$$$$$$$$$$$$$$$$$$$$$$$$$$$$$$$$$$$$$$$$$$$
%Reference Sentence 2:LDU paper
%$$$$$$$$$$$$$$$$$$$$$$$$$$$$$$$$$$$$$$$$$$$$$$$$$$$$$$$$$$$$$$$$$$$$$$$$$$$$$$$$

%Paragraph 1:Linux Scalability의 연구에 대한 설명
%In order to improve Linux scalability, researchers have been optimized memory
%management in Linux by finding and fixing scalability bottlenecks.

%Shared address spaces in multithreaded applications 
%easily become scalability bottlenecks since kernel operations 
%including \code{mmap} and \code{munmap} system calls and \code{page faults}
% handling require per-process locks for synchronization.
%Multithreaded application, for example, can become bottleneck by kernel
%operations on their shared address space, whose operations are the \code{mmap}
%and \code{munmap} system calls and \code{page faults}.
%These operations are synchronized by a single per-process lock.
%BonsaiVM~\cite{AustinTClements2012RCUBalancedTrees} solved this address space
%problem by using the RCU;
%RadixVM~\cite{Clements2013RadixVM} created a new VM using refcache and radix
%tree, which enable \code{munmap}, \code{mmap}, and \code{page fault} on
%non-overlapping memory regions to scale perfectly.
%Alternatively, to avoid contention caused by shared address space locking,
%system programmers change their multithreaded applications to use
%processes~\cite{SilasBoydWickizer2010LinuxScales48}.
 
%Paragraph 2:Fork Scalability에 대한 설명
%Multi-processing environment 
%suffers from scalability bottlenecks due to the well-known fork
%scalability problem~\cite{Andi2011adding}~\cite{Tim2013adding}.
%When Linux spawns child process, the Linux substantially performs locks because
%of protecting the reverse mapping data structure naturally causing
%bottlenecks.
%Oplog~\cite{SilasBoydWickizerPth}, which is an important basis of our approach,
%solves this problem using time-stamp log and
%per-core processing.


%\subsection{Locking}
%Paragraph 1: Locks에 대한 설명
%MCS~\cite{MellorCrummey91}, a scalable exclusive lock, is used in the Linux
%kernel~\cite{MCSLocksKernel}.
%to avoid unfairness at high contention levels, so this scalable exclusive
%lock can be used for fine grained locking in Linux. 
%However, in MCS, since only one thread may hold the lock at a time, it can
%cause low scalability in case of long critical regions.
%Reader-writer lock~\cite{Courtois71} allows either number of readers to execute
%concurrently or single writer to execute.
%Thus, readers-writer locks allow better scalability in case of read-mostly 
%objects.

%In read-mostly data structures, RCU~\cite{McKenney98} can be quite useful
%since it allows read operations to proceed without read locks, and delays
%freeing of data structures to avoid races. One drawback is that as the update
%rate increases, their performance and scalability decrease due to a single
%writer and their synchronization function.
%Consequently, 
%scalable exclusive lock, reader-writer lock
%and RCU require serialization for updates and thus show significant limitation
% on scalability. 

%\subsection{Non-Blocking algorithm}

%Paragraph 1: Lock-free 방법 설명
%One method for the concurrent update is using the non-blocking
%algorithms~\cite{Harris2001Lockfree}~\cite{Fomitchev2004Lockfree}~\cite{Timnat2012},
% which are based on CAS.
%In non-blocking algorithms, each core tries to read the values of shared
%data structures from its local location, but has possibility of reading 
%obsolete values.
%CAS is performed at the time of reading values that are not the current values
%and CAS fails and requires retrials sometimes when the values have been
% overwritten.
%These algorithms execute optimistically as though they read the value at
%location in their data structure;they may obtain stale data at the time.
%When they observed against the current value, they execute a CAS to compare the
%against value.
%The CAS fails when the value has been overridden, and they must be
%retried later on.
%Consequently, both repeated CAS operation and their iteration loop caused by
%CAS fails cause bottlenecks due to inter-core communication
% overheads~\cite{SilasBoydWickizerPth}.
%Moreover, none of the non-blocking algorithms implements an iterator, whose
%data structure just consists of the insert, delete and contains
%operations~\cite{petrank2013lock}.
%The Linux, however, commonly uses the iteration to read, so when applying
%non-blocking algorithms to the Linux, they may meet this iteration problem.
%Petrank~\cite{petrank2013lock} solved this problem by using a consistent
%snapshot of the data structure; this method, however, may require a lot of
%effort to apply its sophisticated algorithms to Linux.
%For evaluation purposes, we implemented Harris linked
%list~\cite{Harris2001Lockfree} to Linux, and we sometimes have failure where
%reading the pointer that had been deleted by updater concurrently result of the
%problem of the iteration.

%Paragraph 2: Linux llist 설명
%Linux kernel uses lock-less list("lock-less NULL terminated single
%list") that are widely used in the Linux kernel to improve scalability.
%In order to delete operation for multiple consumer, the existing
%algorithms traverse the list from beginning or their optimized point.
%On the other hand, lock-less list inserts node at the first of the list, so
% when the CAS operation fails, they will minimally traverse from the head
% node.
%Although lock-less list uses a non blocking method, they retries minimally
%and thus they can significantly reduce inter-core communication bottleneck and
% the repeated loop bottleneck.
%Our proposed method uses this feature in case of inserting the operation log.

%\subsection{Concurrent Update}
%Paragraph 1: Update Rate
%Though sufficient level of performance scalability has been achieved for 
%reader intensive operations through RCU and Hazard pointer, 
%solutions to scalability for update-heavy operations has not been satisfiable.
%A recent paper by Arbel and Attiya~\cite{Arbel2014ConcurrentRCU} shows a new
%design of concurrent search tree called the Citrus tree. The Citrus tree
%combines RCU and fine-grained locks, and it supports concurrent write
%operations that traverse the search tree by using RCU concurrently.
%When increasing the update rate, Citrus tree still suffers from bottlenecks.
%RLU~\cite{Matveev2015RLU} presents a new synchronization mechanism that allows
%unsynchronized sequences of reads to execute concurrently with updates.
%In high update rate, Oplog can achieve substantially multi-core scaling for
%update-heavy data structures.
%Our work focus on update-heavy data structures and uses non-blocking method to
% store the operation log instead of per-core processing.
